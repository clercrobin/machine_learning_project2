\documentclass[10pt,conference,compsocconf]{IEEEtran}

\usepackage{hyperref}
\usepackage{graphicx}	% For figure environment
\usepackage{color}
\usepackage{natbib}
\usepackage{float}
\usepackage{cprotect}
\usepackage{tabularx}
\usepackage{multirow}

\begin{document}
\title{Text analysis: Machine Learning Project 2}

\author{
  Robin Clerc, Pierre Vigier, Jacob Levy Abitbol\\
  \textit{Master of Computer Science, EPFL, Switzerland}
}

\maketitle

\begin{abstract}
Twitter users produce more than 500 million tweets per day since many years, leading to a huge database of text messages of less than 140 symbols (280 since two months). Those text messages include some usual words, abbreviations due to the size constraints, as well as new invented buzz words, user tagging, urls, hashtags or even smileys which are often used to express sentiments. We give here an overview of our work in order to predict if a given tweet is more likely to contain a ":)" or a ":(" smiley with two very different approaches : the first one is a traditional approach based on word embeddings and especially the glove model and the second one : TODO JACOB.
\end{abstract}

\section{Introduction}

Everyday people write some text content on the internet, a crucial task is to detect whether it is a positive or negative message, especially for commercial use or rating purpose. A positive message is very highly correlated with a ":)" smiley and similarly, a negative message is very highly correlated with a ":(" smiley. This assumption makes it far more easier to tag tweets in order to build a training and testing base to train and test a supervised classification network, but means that we are not totally sure to do real sentiment analysis. Anyway, deciding whether a message is positive or negative is often a very subjective task ; even the buildings of the datasets are controversial topics

\section{Models and Methods}
\label{sec:structure-paper}

\subsection{Data Exploration and Preprocessing : text normalization}

We are provided with a training dataset of 2.5 million labeled tweets. One half (1.25 million) contain ":)" and are tagged +1, and the other half contain ":(" and are tagged -1.
The test database contains 10000 unlabeled tweets.

Twitter is a very young and fast social media, with a huge constraint on the number of symbols : 140 at the time of those tweets. It leads to a totally different kind of texts, words from books, usual subject of natural language processing. The constraint over the number of symbols is one of the main reasons leading to many abbreviations, but in our opinion, the biggest problem comes from the huge variety in people writing, without paying attention to their orthography, syntax, or even vocabulary which lead to several writings for the same word, sometimes very distant from each other. The first step will be to normalize those texts at a word level to reduce the word space and perform a kind of clustering of those unusual words.

By reading a few tweets we can observe the major differences between those tweets and the usual corpus.

An important note on this preprocessing is that in this embedding method, we are very likely to use the pretrained word embeddings given by stansford NLP section, trained on billions of tweets, we want our preprocessing to be as close as possible as them, to have a very high correlation between their vocabulary and ours. We found a short ruby program doing this preprocessing, and coded it in python, but some of the parts were impossible due to the preprocessed tokenization.

\begin{itemize}
    \item Smileys such as \^\^  which can be highly correlated to the target labels. However as they are punctuation signs, they are separated by a space due to the preprocessing included in the datasets of this project
    \item Huge amount of typos (for example piercin -> piercing) that can be found thanks to huge dictionaries found on the internet combined to a correction based on edit distance (insertion, deletion, substitution) but which was too slow on the amount of data we have.
    \item Hashtags splitting : We simply removed the \# symbol, we wanted to split those sequences of letters on majuscules as they are more likely to begin a new word, unluckily it was impossible after the initial preprocessing of the project. Note that it was one of the steps of the Stanford preprocessing.
    \item Letter repetitions in some words such as $honeeeyyy$ we want to remove the letters repeated more than 3 times and limit this repetition to 2 times, because in usual words, no letter can be repeated 3 times but can be repeated 2 times. Moreover, as dictionaries cannot easily deal with random number of repetitons of a letter, we made the assumption that two times was the usual typo included in dictionaries
\end{itemize}

We also add the usual processing features : 
\begin{itemize}
    \item Expanding the usual contractions thanks to a dictionary
    \item Replace abbreviation thanks to a dictionary, some of them being specific to twitter.
\end{itemize}

Note that we also did not considered at all some methods used for usual text analysis such as Part-Of-Speech tagging because of our dataset. Moreover, removing the stop words was decreasing our accuracy.

The order has an importance, for example we want to remove the letter repetitions before checking in a dictionary as previously stated.

\subsection{Tweet representation}

Thanks to the preprocessing, we have got a clean tweet, that has a far smaller vocabulary than the initial tweets. One of the main constraints to use neural networks for classification is to have a fixed input size. Thanks to the keras model we can easily build those sequences. We set the limit to 35 words as after the preprocessing, the majority of the tweets are under this limit. We do not want to set it to the maximum length of any tweet because it would lead to too many tweets with much zero padding resulting in very low performance.

Then we inspired ourselves of Glove and Word2Vec and began all our models with an embedding layer which will learn a vector representation of our words.

Our baseline was done without any initialization.

Then we initialized it with the Glove trained on 2 billions of tweets from Stanford University, but let the possibility to this layer to train, because thanks to it we would get both a general information about the meaning of this particular word, and a specialization on our dataset. It is also necessary as some of our words were not found in the glove model.


TODO table of comparison


\subsection{Pre modeling}

An important step due to the order of the training set is to shuffle the whole training base and to remember this order.

\subsection{Models}

The majority of the neural network models about natural language processing make a intensive use of convolutional neural networks. It makes sense because group of words have usually more impact than words, and the position in the tweet of those group of words do not often matter.
As in the lecture and the whole litteracy on the internet, we add a max pooling layer (subsampling) right after.

It is also interesting to stack those convolutional layers as group of words can also have interactions and a meaning.

We end our networks by a fully connected layer and a final layer with a single neuron performing the classification, that is to say returning a value between 0 and 1 which can also be interpreted as a confidence, even if we have a sigmoid the last layer.

As we are performing a binary classification, we set the loss function to binary crossentropy. We also use the Adam optimizer, described as very efficient for binary classification.

For the training, we observed than epochs after the $2^{nd}$ do not add much value to our accuracy.

\subsection{Mix our models}

As many of our models had reasonnably high but different results, we thought than an average of those models would perform best than every single model. Instead of a simple average, we chose to implement a classifier, and the very common AdaBoostClassifier was sufficient for this purpose.

\end{document}
